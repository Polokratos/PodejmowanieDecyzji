\documentclass{article}
\usepackage{graphicx}
\usepackage{polski}
\usepackage[utf8]{inputenc}
\usepackage{amsfonts}
\usepackage{listings}
\usepackage{amssymb}
\usepackage{amsmath}
\usepackage{breqn}

\author{Piotr Koproń \and Bartosz Słupik}
\date{2023.03.03}
\title{Podejmowanie decyzji - projekt zaliczeniowy; wersja 0.1.0}

\begin{document}

\maketitle
\newpage

\section{Wstęp}
Specyfikacja projektu zaliczeniowego na pzedmiot "Podejmowanie Decyzji".
\section{Część specyfikacyjna}
\subsection{Zakres projektu}
\subsubsection{Cele projektu}
Podstawowym celem projektu jest przygotowanie aplikacji do grupowego podejmowania decyzji metodą AHP (metodą porównywania parami). \\
\begin{itemize}
\item aplikacja powinna wykorzystywać metodę EVM i/lub GMM do tworzenia rankingów oraz odpowiednie metody AIJ i/lub AIR do grupowania osądów (macierzy porównywania parami) pochodzących od różnych ekspertów. 
\item aplikacja powinna - pozwalać na używanie tzw. skali fundamentalnej ale również na definiowanie swojej własnej skali porównań przez użytkownika, w tym bezpośrednio skali numerycznej. 
\item aplikacja powinna działać w architekturze klient-serwer 
\item aplikacja powinna umożliwiać pełny eksport danych rankingowych (wszystkich danych wejściowych) do pliku JSON (przykładowy plik załączony w sekcji Ramowy Opis Projektu).
\end{itemize}
\subsubsection{Przykładowy Use Case, który powinien być możliwy do zrealizowania.}
\begin{itemize}
\item Facylitator
	\begin{itemize}
        \item Tworzy nowy ranking
		\begin{itemize}
            \item Definiuje nowe alternatywy
            \item Definiuje kryteria
            \item Definiuje pod-kryteria (jeśli to konieczne)
            \item Definiuje uczestników rankingu
            \item Określa parametry rankingu takie jak:
			\begin{itemize}
                \item Sposób liczenia rankingu (EVM / GMM etc).
                \item Czy ranking ma być kompletny czy nie kompletny
                \item Skale pomiarową
                \item Sposób liczenia niespójności rankingu
                \item Kolejność zadawanych pytań (losowa / konkretna)
                \item Datę i czas od której ranking ma być dostępny
                \item Datę i czas do której ranking ma być dostępny
			\end{itemize}
            \item Rozsyła zaproszenie do udziału w rankingu ekspertom (uczestnikom rankingu)
		\end{itemize}
	\end{itemize}
\item    Eksperci
	\begin{itemize}
        \item Odpowiadają na pytania o porównanie parami alternatyw (na jednym ekranie powinno znaleźć się jedno pytanie (jedno porównanie parami dwóch alternatyw)
        \item Odpowiadają na pytania o porównanie parami kryteriów (podobnie na jednym ekranie jedno porównanie)
        \item "Naciskają" guzik [submit], informujący system, że z ich strony ocena się zakończyła.
	\end{itemize}
\item    Facylitator
	\begin{itemize}
        \item Sprawdza zebrane wyniki
        \item Nadzoruje wykonanie rankingu
        \item Rozsyła wyniki uczestnikom/ekspertom procesu (oraz decydentom zewnętrznym).
        \item Eksportuje wszystkie dane procesu do formatu JSON. 
	\end{itemize}
\end{itemize}
\subsection{Architektura}
\begin{itemize}
	\item Serwer: REST-based C\# TU WSTAW RESZTE TECH STACKU SERWERA
	\item Klient: React+Next.js+TypeScript - zgodnie z mockami w osobnych dokumentach
	\item Wewnętrzne REST API - zgodnie z osobnym dokumentem
	\item Baza danych - zgodnie z osobnym dokumentem
\end{itemize}
\subsection{Zarządzanie testowaniem}
BLOCKED BY: Brak zatwierdzenia specyfikacji przez Klienta 
\subsection{Zarządzanie ryzykiem}
Podstawowym źródłem ryzyka są nieprzewidziane okoliczności. Zaleczne ostrożne estymaty czasu pracy nad funkcjonalnościami.
\section{Część projektowa}
\end{document}