\documentclass[12pt, a4paper]{article}
\usepackage{polyglossia}
\usepackage{hyperref}
\setmainlanguage{polish}
\hypersetup{
  colorlinks,
  citecolor=violet,
  linkcolor=red,
  urlcolor=blue}

\title{Metody i algorytmy podejmowania decyzji - case study}
\author{Bartłomiej Słupik \and Piotr Koproń}

\begin{document}
    \maketitle
    \section*{Case study 1: Jogurty}
    Dane jest 7 rodzajów jogurtów naturalnych dostępnych w sklepie Carrefour:
    \begin{itemize}
        \item Krasnystaw Calpro
        \item Bakoma Bio 
        \item Zott Primo
        \item Maluta Śmiedankowy bez laktozy
        \item Activia
        \item Mleczny Przystanek
        \item Piątnica Skyr
    \end{itemize}

    Celem zadania było zaproponowanie kilku kryteriów porównawczych oraz wyłonienia najlepszego produktu
    dla diety studenta.

    \subsubsection*{Dane}
    Dane dotyczące produktów pobraliśmy ze strony \\ \href{https://www.carrefour.pl/mleko-nabial-jaja/jogurt-kefir-maslanka-serki-homogenizowane/jogurty-naturalne}{https://www.carrefour.pl/mleko-nabial-jaja/jogurt-kefir-maslanka-serki-homogenizowane/jogurty-naturalne}
    

    \subsubsection*{Kryteria}
    Zaproponowaliśmy następujące 5 kryteriów:
    \begin{itemize}
        \item Cena
        \item Objętość
        \item Smak
        \item Zawartość białka w 100g
        \item Zawartość cukru w 100g
    \end{itemize}

    \subsubsection*{Eksperci}
    W ankiecie wzięliśmy udział my sami (w dwójkę), gdyż stanowimy reprezentatywną grupę dla tematu i nie lobbujemy za żadną z opcji.

    \subsubsection*{Algorytm}
    Do agregacji danych użyliśmy metody AIJ, a do obliczania rankingu - metody EVM.

    \subsubsection*{Wyniki}
    Macierz wynikowa prezentuje się następująco:
    \begin{table}[h]
        \centering
        \begin{tabular}{rl}
            Krasnystaw & 0.12 \\
            Bakoma Bio & 0.11 \\ 
            Zott Primo & 0.17 \\ 
            Maluta     & 0.13 \\
            Activia    & 0.10 \\
            Mleczny Przystanek & 0.16 \\
            Piątnica Skyr & 0.21 \\
        \end{tabular}
    \end{table}

    Wygrała zatem Piątnica, gdyż mimo wyższej ceny ma dużo lepszy skład. Drugie miejsce zajął Zott, który łączy przystępną
    cenę i niską zawartość cukru.

    \pagebreak

    \section*{Case study 2: Olimpiada}
    Celem zadania było przeprowadzić ranking możliwych organizatorów Igrzysk Olimpijskich w 2036r.
    Dane są trzy kandydatury:
    \begin{itemize}
        \item Meksyk
        \item Indonezja
        \item Turcja
    \end{itemize}

    \subsubsection*{Kryteria}
    \begin{itemize}
        \item PKB per capita
        \item Dostępność połączeń lotniczych
        \item Atrakcyjność ogólna
        \item Bezpieczeństwo
        \item Doświadczenie w organizacji wielkich imprez
        \item Pogoda w miesiącu zawodów
    \end{itemize}

    \subsubsection*{Eksperci}
    W ankiecie wzięło udział 4 ekspertów. Ekspert nr 1 lobbuje za Turcją, a ekspert nr 2 - za Indonezją.
    W tym celu wybierają po trzy kryteria, które mają szanse zostać uznane za najważniejsze przez innych.
    Pod tymi kryteriami dają nienaturalne noty. W pozostałych kryteriach oceniają obiektywnie.

    \subsubsection*{Algorytm}
    Do agregacji danych użyliśmy metody AIJ, a do obliczania - metody GMM.

    \subsubsection*{Wyniki}
    Macierz wynikowa prezentuje się następująco:
    \begin{table}[h]
        \centering
        \begin{tabular}{rl}
            Meksyk & 0.28 \\ 
            Indonezja & 0.33 \\
            Turcja & 0.37
        \end{tabular}
    \end{table}

    Widać, że lobbystom udało się zagwarantować wysokie pozycje swoim krajom.
\end{document}
